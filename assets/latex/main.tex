\documentclass[12pt]{article}
\usepackage{tikz}
\usepackage{subfig}
\usepackage{graphicx}
\usepackage{xcolor}
\usepackage[margin=1.0in]{geometry}
%\pagestyle{empty}

\definecolor{mygray}{rgb}{0.8, 0.8, 0.8} % Light gray

\usepackage{fancyhdr}
\usepackage{graphicx}
% todo
\setlength{\headheight}{65.90973pt}  % Nastavenie výšky hlavičky
% \setlength{\headsep}{15pt}        % Posun hlavičky od textu
\addtolength{\topmargin}{5pt} % Kompenzácia horného okraja
\pagestyle{fancy}

\fancyhead[L]{ % Logo on the left
   % \hspace{1cm} % Add horizontal space to push the logo to the right
    % \parbox{\textwidth}{
    \raisebox{1ex}{\includegraphics[width=2.1cm]{assets/latex/stu-fei-vector-logo.png}}
    % }
}

\usepackage{qrcode}

\fancyhead[R]{
    \raisebox{5ex}{
        \scalebox{0.5}{% Zmenši QR kód na 50 % pôvodnej veľkosti
           \qrcode{ {{.QrCode}} }
        }
    }
}

% \fancyhead[C]{ % Centered text with details
%     \parbox{1\textwidth}{ % Adjusted width
%         \centering % Center alignment
%         \textbf{Meno: Jozko Mrkvicka} \hspace{2em}
%         \textbf{ID: 120120} \\
%         \textbf{Dátum: \today} \hspace{2em}
%         \textbf{Miestnosť: CD300} \hspace{2em}
%         \textbf{Čas: 10:00}
%     }
% }




\fancyhead[C]{ % Centered text with details
    \parbox{1\textwidth}{ % Adjusted width
        \centering % Center alignment
        \textbf{Meno: {{.Meno}}} \hspace{2em}
        \textbf{ID: {{.ID}}} \\ \hspace{2em}
        \textbf{Dátum: {{.Datum}}} \hspace{2em}
        \textbf{Miestnosť: {{.Miestnost}}} \hspace{2em}
        \textbf{Čas: {{.Cas}}}
    }
}

% \fancyfoot[C]{\thepage} % Číslo strany v päte v strede


%% DOBRY KOD
% \begin{document}
% \newcounter{rowCounter} % Create a new counter for rows

% \begin{center} 
%         % Function to generate rows for each block
%         \def\generateBlock#1#2{
%             \draw[thick, black] (-1, -22) rectangle (11.5, -0.5);
            
%                \foreach \position/\letter in {1/A, 2/B, 3/C, 4/D, 5/E} {
%             \node at ({\position * 2}, -1.5) {\letter};
%         }
        
%             \foreach \line in {#1,...,#2} {
%                 \stepcounter{rowCounter} % Increment the counter
%                 \begin{scope}[yshift=-\the\value{rowCounter}cm] % Position each row
%                     \node at (0,-1.5) {\normalsize\textbf{\line}}; % Display row number
%                     \foreach \position in {1,2,3,4,5} { 
%                         \node[draw,rectangle,inner sep=8pt] at ({\position * 2},-1.5) {};
%                     }
%                 \end{scope}
                
%                 % Optional: Draw separator lines every 5 rows
%                 \ifnum\value{rowCounter}=5 \draw[thin, black] (-1,-\the\value{rowCounter} cm-2cm) -- (11.5,-\the\value{rowCounter} cm-2cm);\fi
%                 \ifnum\value{rowCounter}=10 \draw[thin, black] (-1,-\the\value{rowCounter} cm-2cm) -- (11.5,-\the\value{rowCounter} cm-2cm);\fi
%                 \ifnum\value{rowCounter}=15 \draw[thin, black] (-1,-\the\value{rowCounter} cm-2cm) -- (11.5,-\the\value{rowCounter} cm-2cm);\fi
%             }
%         }
% \begin{tikzpicture}[font=\small]
%         % Block 1 (1-20)
%         \generateBlock{1}{20}
% \end{tikzpicture}   
% \begin{tikzpicture}[font=\small]
%         % Reset counter     
%         \setcounter{rowCounter}{0}
%         % Block 2 (21-40)
%         \generateBlock{21}{40}
%     \end{tikzpicture}
%     \begin{tikzpicture}[font=\small]
%         % Reset counter 
%         \setcounter{rowCounter}{0}
%         % Block 3 (41-60)
%         \generateBlock{41}{60}
%     \end{tikzpicture}
% \end{center}


% \end{document}


%%% tiez dobry kod zautomatizovany
% \usepackage{pgf} % For pgfmath calculations

% % Define counters
% \newcounter{rowCounter} % This will continuously count rows across block
% \setcounter{rowCounter}{1}
% \newcounter{rowPosition} % This will reset to 1 for each block
% \newcounter{blocksNeeded} % To store the number of blocks needed
% \newcounter{remainingQuestions} % Remaining questions to be processed

% % Ceiling division function to calculate the number of blocks needed
% \newcommand{\ceilDiv}[1]{%
%     \setcounter{blocksNeeded}{\numexpr(#1 + 19) / 20\relax} % Use integer arithmetic for ceiling division
% }

% % Function to generate each block with a specified range of question numbers
% \newcommand{\generateBlock}[2]{ % Takes `start` and `end` question numbers
%     \setcounter{rowPosition}{1} % Reset rowPosition for each new block
%     \begin{tikzpicture}[font=\small]
%         \draw[thick, black] (-1, -22) rectangle (11.5, -0.5); % Outer rectangle

%         % Header labels
%         \foreach \position/\letter in {1/A, 2/B, 3/C, 4/D, 5/E} {
%             \node at ({\position * 2}, -1.5) {\letter};
%         }

%         % Generate rows from `#1` (start) to `#2` (end)
%         \foreach \line in {#1,...,#2} {
%             \begin{scope}[yshift=-\the\value{rowPosition}cm] % Adjust row position
%                 \node at (0, -1.5) {\normalsize\textbf{\the\value{rowCounter}}}; % Display question number as integer
%                 \foreach \position in {1,2,3,4,5} {
%                     \node[draw, rectangle, inner sep=8pt] at ({\position * 2}, -1.5) {}; % Draw boxes
%                 }

%                 \stepcounter{rowPosition} % Increment rowPosition within the block
%                 \stepcounter{rowCounter} % Increment rowCounter across all blocks
%             \end{scope}
%         }
%     \end{tikzpicture}
% }

% % Function to generate multiple blocks based on the total number of questions
% \newcommand{\generateMultipleBlocks}[1]{ % Total number of questions as input
%     \setcounter{remainingQuestions}{#1} % Initialize remaining questions
%     \ceilDiv{#1} % Calculate blocks needed
    
%     % Generate blocks as per calculated blocksNeeded
%     \foreach \blockIndex in {1,...,\the\value{blocksNeeded}} {
%         % Determine the start and end numbers for this block
%         \pgfmathsetmacro{\startQuestion}{\the\value{rowCounter}}
%         \pgfmathsetmacro{\endQuestion}{min(\the\value{rowCounter} + 19, #1)}

%         % Convert startQuestion and endQuestion to integer to avoid decimal display
%         \generateBlock{\number\startQuestion}{\number\endQuestion}

%         % Decrement remaining questions by the number of questions in this block
%         \addtocounter{remainingQuestions}{-\the\value{rowPosition}}
%     }
% }

% \begin{document}

% \begin{center}
%     \generateMultipleBlocks{60}
% \end{center}

% \end{document}


\usepackage{pgf}

% Define counters
\newcounter{rowCounter} % This will continuously count rows across blocks
\setcounter{rowCounter}{1}
\newcounter{rowPosition} % This will reset to 1 for each block
\newcounter{blocksNeeded} % To store the number of blocks needed
\newcounter{remainingQuestions} % Remaining questions to be processed

% Ceiling division function to calculate the number of blocks needed
\newcommand{\ceilDiv}[1]{%
    \pgfmathsetmacro{\tempResult}{int((#1 + 19) / 20)} % Calculate ceiling division
    \setcounter{blocksNeeded}{\tempResult} % Store result in blocksNeeded
    %\theblocksNeeded % Output the result
}

% Function to generate each block with a specified range of question numbers
\newcommand{\generateBlock}[2]{ % Takes `start` and `end` question numbers
    \setcounter{rowPosition}{1} % Reset rowPosition for each new block

    \begin{tikzpicture}[font=\small]
        \draw[thick, black] (-1, -22) rectangle (11.5, -2); % Outer rectangle


        % Header labels
        \foreach \position/\letter in {1/A, 2/B, 3/C, 4/D, 5/E} {
            \node at ({\position * 2}, -1.5) {\letter};
        }

        % Generate rows from `#1` (start) to `#2` (end), but limit to the exact total number of questions
        % todo - pridanie premennej co by prenasobila yshift tak, aby boli vacsie medzeri medzi riadkami, a tym padom lepsie by sa detekovali kruzky
        \foreach \line in {#1,...,#2} {
            \begin{scope}[yshift=-\the\value{rowPosition}cm - 1.5cm] % Adjust row position
                \node at (0, 0) {\normalsize\textbf{\the\value{rowCounter}}}; % Display question number as integer
                \foreach \position in {1,2,3,4,5} {
                    \node[mygray] at ({\position * 2},0) {\scalebox{2}{\textbf{$\times$}}};
                    \node[draw, rectangle, inner sep=5pt] at ({\position * 2}, 0) {}; % Draw boxes
                }            
        \ifnum\value{rowPosition}=5
            \draw[thin, black] (-1, -\the\value{rowPosition} cm + 4.5cm) -- (11.5, -\the\value{rowPosition} cm + 4.5cm);
        \fi
        \ifnum\value{rowPosition}=10
            \draw[thin, black] (-1, -\the\value{rowPosition} cm + 9.5cm) -- (11.5, -\the\value{rowPosition} cm + 9.5cm);
        \fi
        \ifnum\value{rowPosition}=15
            \draw[thin, black] (-1, -\the\value{rowPosition} cm + 14.5cm) -- (11.5, -\the\value{rowPosition} cm + 14.5cm);
        \fi
                \stepcounter{rowPosition} % Increment rowPosition within the block
                \stepcounter{rowCounter} % Increment rowCounter across all blocks
                
            \end{scope}
        }
    \end{tikzpicture}
}

% Function to generate multiple blocks based on the total number of questions
\newcommand{\generateMultipleBlocks}[1]{ % Total number of questions as input
    \setcounter{remainingQuestions}{#1} % Initialize remaining questions
    \ceilDiv{#1} % Calculate blocks needed
    
    % Generate blocks as per calculated blocksNeeded
    \foreach \blockIndex in {1,...,\the\value{blocksNeeded}} {
        % Determine the start and end numbers for this block
        \pgfmathsetmacro{\startQuestion}{\the\value{rowCounter}}
        \pgfmathsetmacro{\endQuestion}{min(\the\value{rowCounter} + 19, #1)}

        % Convert startQuestion and endQuestion to integer to avoid decimal display
        \generateBlock{\number\startQuestion}{\number\endQuestion}

        % Decrement remaining questions by the number of questions in this block
        \addtocounter{remainingQuestions}{-\the\value{rowPosition}}
    }
}

\begin{document}

\begin{center}
    \generateMultipleBlocks{45}
\end{center}

\end{document}
